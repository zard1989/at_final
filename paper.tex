\documentclass[12pt]{article}

% This first part of the file is called the PREAMBLE. It includes
% customizations and command definitions. The preamble is everything
% between \documentclass and \begin{document}.

\usepackage[margin=1in]{geometry}  % set the margins to 1in on all sides
\usepackage{graphicx}              % to include figures
\usepackage{amsmath}               % great math stuff
\usepackage{amsfonts}              % for blackboard bold, etc
\usepackage{amsthm}                % better theorem environments
\usepackage{amssymb}


% various theorems, numbered by section

\newtheorem{thm}{Theorem}[section]
\newtheorem{lem}[thm]{Lemma}
\newtheorem{prop}[thm]{Proposition}
\newtheorem{cor}[thm]{Corollary}
\newtheorem{conj}[thm]{Conjecture}
\newtheorem{defn}[thm]{Definition}


\DeclareMathOperator{\id}{id}

\newcommand{\bd}[1]{\mathbf{#1}}  % for bolding symbols
\newcommand{\RR}{\mathbb{R}}      % for Real numbers
\newcommand{\ZZ}{\mathbb{Z}}      % for Integers
\newcommand{\col}[1]{\left[\begin{matrix} #1 \end{matrix} \right]}
\newcommand{\comb}[2]{\binom{#1^2 + #2^2}{#1+#2}}
\newcommand{\vect}[1]{\vec{#1}}


\begin{document}


\nocite{*}

\title{Algebraic Topology Final Report: Morse Theory}

\author{Shih-Kai Chiu \\
r01221031}

\maketitle

\begin{abstract}
  This report covers roughly the part I of J. Milnor's book(citation). The main
  result is the theorem of Morse: every smooth manifold has the same homotopy
  type of a CW-complex. We prove this theorem by developing the concept of
  non-degenerate smooth functions, their index at critical points and the
  correspondence between the index and the dimension of attaching cells. After
  the main theorem, the existence of non-degenerate smooth functions on
  manifolds is established, and as an application, an alternating proof of the
  Lefschetz hyperplane theorem is given.
\end{abstract}


\section{Basic Definitions and Lemmas}


\subsection{Attaching Cells}

Let $Y$ be any topological space, and let
\begin{equation}
  e^k = \{ x \in \RR^k | \left| x \right| < 1 \}
\end{equation}
be the k-cell. The boundary
\begin{equation}
  \dot{e}^k = \{ x \in \RR^k | \left| x \right| = 1 \}
\end{equation}
is the (k-1)-sphere, $S^{k-1}$. Specifically, we put $e^0$ as a point and set
$\dot{e}^0 = \emptyset$. We call a continuous map $g: S^{k-1} \to Y$ an
attaching map and obtain a new topological space
\begin{equation}
  Y \cup_g e^k
\end{equation}
by identifying each $x \in \dot{e}^k$ with $g(x) \in Y$.


\subsection{Non-degenerate critical points}

Let $M^n$ be a smooth manifold of dimension $n$, and let $f: M \to \RR$ be a
smooth function on $M$. We recall that a point $x \in M$ is called a critical
point of $f$ if the differential map $ df_p : T_pM \to \RR $ is singular. This
is equivalent to $\frac{\partial f}{\partial x^i} (p) = 0$ for all $i = 1,
.. ,k$ and any local coordinate $(x^1, .., x^k)$, since the map $df_p$ in local
coordinate is a row matrix with a non-empty kernel.

\begin{defn}
  (Non-degenerate critical point) Let $p \in M$ and let $(x^1, .., x^k)$ be a
  local coordinate at $p$. $p$ is called a non-degenerate critical point of $f$
  if
  \begin{equation}
    \left( \frac{\partial^2 f}{\partial x^i \partial x^j} (p) \right)
  \end{equation}
  is an invertible matrix. On the other hand, $p$ is a called degenerate
  critical point if the matrix is singular.
\end{defn}

The definition above is independent of local coordinates. For, if $(x^1, ..,
x^k)$ and $(y^1, .., y^k)$ are two local coordinates at $p$, then
\begin{equation}
      \left( \frac{\partial^2 f}{\partial y^k \partial y^l} (p) \right) =
      J^T\left( \frac{\partial^2 f}{\partial x^i \partial x^j} (p) \right)J,
\end{equation}
where $J = \left( \frac{\partial x^i}{\partial y^j}(p) \right)_{i,j}$ is the
coordinate change matrix, which is non-singular.

\section{Homotopy Type in Terms of Critical Values}


\section{Manifolds in Euclidean Space: The Existence of Non-degenerate
  Functions}

Let $M$ be a smooth manifold of dimension $k$. By Whitney Embedding Theorem, we
can embed $M$ into $\RR^n, n > k$. Therefore, without loss of generality we may
assume $M \subset \RR^n$. For each $p \in \RR^n$, define a smooth function $L_p
: M \to \RR $ by
\begin{equation}
  L_p(q) = |q-p|^2 = (q-p) \cdot (q-p) = q\cdot q-2q \cdot p + p \cdot p
\end{equation}

We want to find a $p \in \RR^n$ such that $L_p$ has no degenerate critical
points in $M$. Denote $L_p$ as $f$. Let $q \in M$ and let $(u^1, .., u^k)
\mapsto (x^1(u), .., x^n(u))$ be a local coordinate centered at $q$. Then $q$ is
a critical point if $\frac{\partial f}{\partial u^i}(q) = 0$ for all $i = 1, ..,
k.$; that is,
\begin{equation}
  \frac{\partial \vect{x}}{\partial u^i}(q) \cdot (q-p) = 0.
\end{equation}
It follows that $q \in M$ is a critical point of $L_p$ if and only if $p$ lies
in a line perpendicular to $M$ at $q$. Differentiate again, and we get
\begin{equation}
  \frac{\partial^2 f}{\partial u^i \partial u^j}(q) = 
  2\left(
    \frac{\partial^2 \vect{x}}{\partial u^i \partial u^j}(q) \cdot (q-p) +
    \frac{\partial \vect{x}}{\partial u^i}(q) \cdot \frac{\partial \vect{x}}{\partial u^j}(q)
  \right)
\end{equation}

We get the following conclusions:


$L_p$ is non-degenerate if and only if

\begin{itemize}
\item For each critical point $q$, $p$ lies in a line perpendicular to $M$ at
  $q$.
\item the matrix $\left(
    \frac{\partial^2 \vect{x}}{\partial u^i \partial u^j}(q) \cdot (q-p) +
    \frac{\partial \vect{x}}{\partial u^i}(q) \cdot \frac{\partial \vect{x}}{\partial u^j}(q)
  \right)$
  is non-singular.
\end{itemize}

Thus the main purpose of this section is to establish the existence of such $p
\in \RR^n$.

\subsection{Focal Points}

Let $N = \{ (q,v) \in \RR^{2n} | q \in M, v \in (T_qM)^{\perp} \subset \RR^n \}$
be the normal bundle of $M$ in $\RR^n$. Then $N$ is a manifold of dimension
$k+(n-k)=n$. We define a map $E$ that maps $N$ into the ambient space $\RR^n$ by
\begin{equation}
  E(q, v) = q + v.
\end{equation}
This is a differentiable map between two manifolds of dimension $n$. The point
$q+v$ thus lies in a normal line that is perpendicular to $M$ at $p$. This leads
to the definition of focal points:

\begin{defn}
  $e \in \RR^n$ is a focal point of $(M, q)$ with multiplicity $\mu > 0$ if $e =
  q+v$, where $(q, v) \in N$ and the Jacobian of $E$ at $(q, v)$ has nullity
  $\mu > 0$. $e$ is called a focal point of $M$ if $e$ is a focal point of $(M,
  q)$, $q \in M$.
\end{defn}

Next we fix a $\vect{q} \in M$ and a unit normal vector $\vect{v} \in \RR^n$ and
study the focal points of $(M, q)$ along the line $\vect{l} =
\vect{q}+t\vect{v}$. Before this, we first introduce the second the second
fundamental form of a manifold in $\RR^n$:

\begin{defn}
  Let $\vect{q} \in M$ and let $\vect{v} \in (T_qM)^\perp$. The second
  fundamental form in the direction of $\vect{v}$ is defined as the matrix

  \begin{equation}
    \left(
      \vect{v} \cdot \vect{l}_{ij}
    \right)
  \end{equation}
  where $\vect{l}_{ij} = \frac{\partial^2 \vect{x}}{\partial u^i \partial
    u^j}(\vect{q})$.
  The eigenvalues of the matrix is called the principal curvatures $K_1$, ..,
  $K_k$ of $M$ at $\vect{q}$ in the normal direction $\vect{v}$.
\end{defn}

Now comes the key lemma in this section:

\begin{lem}
  The focal points of $(M, q)$ along $\vect{l}$  are precisely the points
  $\vect{q}+K_i^{-1}\vect{v}$, where $1 \le i \le k$, $K_i \ne 0$. Thus there
  are at most $k$ focal points of $(M,q)$ along $\vect{l}$, each being counted
  with its proper multiplicity.
\end{lem}

\begin{proof}
  First we are going to find a local coordinate of $N$ centered at $(q, v)$. Let
  $\vect{w}_1(u^1, .., u^k)$, .., $\vect{w}_{n-k}(u^1, .., u^k)$ be orthonormal
  vector fields in $N$. Then we have a local coordinate of $N$ at $(q, v)$
  defined by
  \begin{equation}
    (u^1, .., u^k, t^1, .., t^{n-k}) \mapsto (\vect{x}(u),
    \sum_{\alpha=1}^{n-k}t^{\alpha} \vect{w}_{\alpha}(u^1, .., u^k)) \in N
  \end{equation}
  Write $E: N \to \RR$ in the coordinate as $E(u, t) = \vect{e}(u, t) \in \RR^n$
  and get
  \begin{equation}
    \vect{e}(u, t) = \vect{x}(u) + \sum_{\alpha=1}^{n-k}t^{\alpha} \vect{w}_{\alpha}(u^1, .., u^k))
  \end{equation}
  with partial derivatives
  \begin{equation}
    \left\{
      \begin{array}{lcl}
        \frac{\partial \vect{e}}{\partial u^i} & = &
        \frac{\partial \vect{x}}{\partial u^i} + 
        \sum_{\alpha} t^{\alpha}\frac{\vect{w}_{\alpha}}{\partial u^i}  \\
        \frac{\partial \vect{e}}{\partial t^\beta} & = & \vect{w}_\beta
      \end{array}
    \right.
  \end{equation}
  
  Since the n-vectors $\frac{\partial \vect{x}}{\partial u^1}$, ..,
  $\frac{\partial \vect{x}}{\partial u^k}$, $\vect{w}_1$, .., $\vect{w}_{n-k}$
  form a basis of $T_q\RR^n$, the matrix
  \begin{equation}
    A = \left(
      \begin{matrix}
        \left(\frac{\partial \vect{x}}{\partial u^1}\right)^T \\
        \vdots \\
        \left(\frac{\partial \vect{x}}{\partial u^k}\right)^T \\
        \left(\vect{w}_1\right)^T \\
        \vdots \\
        \left(\vect{w}_{n-k}\right)^T
      \end{matrix}
    \right)
  \end{equation}
  is non-singular (we represent a vector as a column matrix). Multiply $A$ with
  the Jacobian $J$ of $E(u, t)$ and we get
  \begin{equation}
    \begin{aligned}
      AJ &= 
      \left(
        \begin{matrix}
          \left(\frac{\partial \vect{x}}{\partial u^1}\right)^T \\
          \vdots \\
          \left(\frac{\partial \vect{x}}{\partial u^k}\right)^T \\
          \left(\vect{w}_1\right)^T \\
            \vdots \\
          \left(\vect{w}_{n-k}\right)^T
        \end{matrix}
      \right)
      \left(
        \begin{matrix}
          \frac{\partial \vect{e}}{\partial u^1} &
          \cdots &
          \frac{\partial \vect{e}}{\partial u^k} &
          \frac{\partial \vect{e}}{\partial t^1} &
          \cdots &
          \frac{\partial \vect{e}}{\partial t^{n-k}} &
        \end{matrix}
      \right) \\
      &=
      \left(
        \begin{matrix}
          \left( \frac{\partial \vect{x}}{\partial u^i} \cdot \frac{\partial
              \vec{x}}{\partial u^j} + \sum_{\alpha}
            t^{\alpha}\frac{\vec{w}_{\alpha}}{\partial u^i} \cdot \frac{\partial
              \vect{x}}{\partial u^j} \right) & 0
          \\
          \sum_{\alpha} t^{\alpha}\frac{\vect{w}_{\alpha}}{\partial u^i} \cdot
          \vect{w}_{\beta} & \id
        \end{matrix}
      \right)
    \end{aligned}
  \end{equation}
  
  Since $J$ has the same rank of $AJ$, $J$ has the same nullity as the upper
  left block of $AJ$. Using the identity
  \begin{equation}
    \begin{aligned}
      0 &= \frac{\partial}{\partial u^i}\left(\vect{w}_{\alpha} \cdot
        \frac{\partial
          \vect{x}}{\partial u^j}\right) \\
      &= t^{\alpha}\frac{\vect{w}_{\alpha}}{\partial u^i} \cdot \frac{\partial
        \vect{x}}{\partial u^j} + \vect{w}_{\alpha} \cdot \frac{\partial^2
        \vect{x}}{\partial u^i \partial u^j},
    \end{aligned}
  \end{equation}
  we see that this upper left hand block is just the matrix 
  \begin{equation}
    \left( g_{ij} - \sum_{\alpha}t^{\alpha}\vect{w}_{\alpha} \cdot \vect{l}_{ij} \right)
  \end{equation}

  We choose the Riemman normal coordinate at $q$, so that $g_{ij}(q) =
  \delta_{ij}$. Now, $\sum_{\alpha}t^{\alpha}\vect{w}_{\alpha} = t\vect{v}$. The
  matrix
  \begin{equation}
    \label{eq:key}
    \left(
      \delta_{ij}-t\vect{v} \cdot  \vect{l}_{ij}
    \right)
  \end{equation}
  is singular if and only if $\frac{1}{t} = K_i$. Furthermore, $\mu$ equals the
  number of distinct $K_i$, each of which counted with its own
  multiplicity. This proves the lemma.
\end{proof}

Suppose $\vect{p} = \vect{q}+t\vect{v}$. Then $\vect{q}$ is a critical point of
$L_p$. Recall the criterion of $q$ being a non-degenerate critical point of
$L_p$. The criterion is exactly Eq (\ref{eq:key}). We thus obtain the
correspondence between the non-degeneracy of $L_p$ and the idea of focal points:

\begin{cor}
  The function $L_p$ is non-degenerate if and only if $p$ is not a focal point
  of $M$. \qed
\end{cor}

$L_p$ is degenerate if and only if $p$ is a focal point, which is a critical
value of a smooth map. By the following theorem of Sard, we see that there are
infinitely many $p \in \RR^n$ such that $L_p$ is non-degenerate:

\begin{thm}[Sard]
  Let $f: M^m \to N^n$ be a smooth map between two smooth manifolds. Then the
  critical values has measure 0 in $N^n$. \qed
\end{thm}

This established the existence of non-degenerate functions on $M$. As a result,
the main theorem in this report is proved:

\begin{thm}
  A differentiable manifold has the homotopy type of a CW-complex.\qed
\end{thm}


The preceding corollary can be sharpened as follows. Let $k \ge 0$ be an integer
and let $K \subset M$ be a compact set.

\begin{cor}
  Any bounded smooth function $f: M \to \RR$ can be uniformly approximated by a
  smooth function $g$ which has no degenerate critical points. Furthermore $g$
  can be chosen so that the i-th derivatives of $g$ on the compact set $K$
  uniformly approximate the corresponding derivatives of $f$, for $i \le k$.
\end{cor}

\begin{proof}
  Choose some embedding $h:M \to \RR^n$ of $M$ as a bounded subset of some
  euclidean space so that the first coordinate $h_1$ is exactly
  $f$. \footnote{By Whitney embedding theorem, there is an embedding $\phi : M
    \in R^n$. Compose each coordinate of $\phi$ with $arctan$, we get a bounded
    embedding, since $\psi = (\tan^{-1} \circ \phi^1, .., \tan^{-1} \circ
    \phi^n)$ is one-to-one and the differential map is injective. Let $h = (f,
    \tan^{-1}\phi)$. Then $h$ is an embedding from $M$ into $\RR^{n+1}$ such
    that $h_1 = f$.} Let c be a large number. Choose a point
  \begin{equation}
    p = (-c+\epsilon_1, \epsilon_2, .., \epsilon_n)
  \end{equation}
  close to $(-c, 0, .., 0) \in \RR^n$ such that the function $L_p: M \to \RR$ is
  non-degenerate. Define
  \begin{equation}
    g(x) = \frac{L_p(x)-c^2}{2c}
  \end{equation}
  A short computation gives
  \begin{equation}
    \begin{aligned}
      g(x) &= \frac{\sum_i h_i^2(x)-h_i(x)p_i+p_i^2-c^2}{2c} \\
           &= f(x) + \frac{\sum_i h_i^2(x)-\sum_i h_i(x)\epsilon_i+\sum_i \epsilon_i^2}{2c}-\epsilon_1.
    \end{aligned}
  \end{equation}
  Since $K$ is compact and each $h_i$ is continuous and bounded, it's easy to
  see that $g(x)$ approximates $f(x)$ uniformly by choosing $c$ large enough and
  $\epsilon_i$ small enough. 
\end{proof}

The preceding theory can also be used to describe the index of the function
$L_p$ at a critical point:

\begin{lem}[Index theorem for $L_p$]
  The index of $L_p$ at a non-degenerate critical point $q \in M$ is equal to
  the number of focal points of (M, q) which lie on the segment from $q$ to $p$;
  each focal point being counted with its multiplicity.
\end{lem}

\begin{proof}
  The index of the matrix 
  \begin{equation}
    \left( \frac{\partial^2 L_p}{\partial u^i \partial
        u^j}(q) \right) = 2(g_{ij}-t\vect{v} \cdot \vect{l}_{ij})
  \end{equation}
  equals the number of negative eigenvalues counting with multiplicity. Recall
  that we can choose the normal coordinate such that $(g_{ij})$ is
  identity. Then$\left( \frac{\partial^2 L_p}{\partial u^i \partial u^j}(q)
  \right)$ has a negative eigenvalue if and only if $1-K_it < 0$ for some
  principal curvatures $K_i$. Thus $K_i > \frac{1{t}$. Recall that the focal
    points that lies on the line from $q$ through $p$ are of the form
    $\vect{q}+K_i^{-1}\vect{v}$. Therefore the correspondece established. 
    
\end{proof}

\section{The Lefschetz Theorem on Hyperplane Sections}


\end{document}