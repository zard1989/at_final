% \documentclass[12pt]{article}
\documentclass[a4paper,11pt,reqno]{amsart}

% This first part of the file is called the PREAMBLE. It includes
% customizations and command definitions. The preamble is everything
% between \documentclass and \begin{document}.

\usepackage[margin=1in]{geometry}  % set the margins to 1in on all sides
\usepackage{graphicx}              % to include figures
\usepackage{amsmath}               % great math stuff
\usepackage{amsfonts}              % for blackboard bold, etc
\usepackage{amsthm}                % better theorem environments
\usepackage{amssymb}


% various theorems, numbered by section

\newtheorem{thm}{Theorem}[section]
\newtheorem{lem}[thm]{Lemma}
\newtheorem{prop}[thm]{Proposition}
\newtheorem{cor}[thm]{Corollary}
\newtheorem{conj}[thm]{Conjecture}
\newtheorem{defn}[thm]{Definition}
\newtheorem{asser}[thm]{Assertion}
\newtheorem{rmk}[thm]{Remark}

\DeclareMathOperator{\id}{id}

\newcommand{\bd}[1]{\mathbf{#1}}  % for bolding symbols
\newcommand{\RR}{\mathbb{R}}      % for Real numbers
\newcommand{\ZZ}{\mathbb{Z}}      % for Integers
\newcommand{\col}[1]{\left[\begin{matrix} #1 \end{matrix} \right]}
\newcommand{\comb}[2]{\binom{#1^2 + #2^2}{#1+#2}}
\newcommand{\vect}[1]{\vec{#1}}


\begin{document}


\nocite{*}

\title{Algebraic Topology Final Report: Morse Theory}

\author{Shih-Kai Chiu \\
r01221031}

\maketitle

\begin{abstract}
  This report covers roughly the part I of J. Milnor's book(citation). The main
  result is the theorem of Morse: every smooth manifold has the same homotopy
  type of a CW-complex. We prove this theorem by developing the concept of
  non-degenerate smooth functions, their index at critical points and the
  correspondence between the index and the dimension of attaching cells. After
  the main theorem, the existence of non-degenerate smooth functions on
  manifolds is established, and as an application, an alternating proof of the
  Lefschetz hyperplane theorem is given.
\end{abstract}


\section{Basic Definitions and Lemmas}


\subsection{Attaching Cells}

Let $Y$ be any topological space, and let
\begin{equation}
  e^k = \{ x \in \RR^k | \left| x \right| < 1 \}
\end{equation}
be the k-cell. The boundary
\begin{equation}
  \dot{e}^k = \{ x \in \RR^k | \left| x \right| = 1 \}
\end{equation}
is the (k-1)-sphere, $S^{k-1}$. Specifically, we put $e^0$ as a point and set
$\dot{e}^0 = \emptyset$. We call a continuous map $g: S^{k-1} \to Y$ an
attaching map and obtain a new topological space
\begin{equation}
  Y \cup_g e^k
\end{equation}
by identifying each $x \in \dot{e}^k$ with $g(x) \in Y$.


\subsection{Non-degenerate critical points}

Let $M^n$ be a smooth manifold of dimension $n$, and let $f: M \to \RR$ be a
smooth function on $M$. We recall that a point $x \in M$ is called a critical
point of $f$ if the differential map $ df_p : T_pM \to \RR $ is singular. This
is equivalent to $\frac{\partial f}{\partial x^i} (p) = 0$ for all $i = 1,
.. ,k$ and any local coordinate system $(x^1, .., x^k)$, since the map $df_p$ in
local coordinates is a row matrix with a non-empty kernel.

\begin{defn}
  Let $p \in M$ and let $(x^1, .., x^k)$ be a local coordinate system centered
  at $p$. $p$ is called a non-degenerate critical point of $f$ if
  \begin{equation}
    \left( \frac{\partial^2 f}{\partial x^i \partial x^j} (p) \right)
  \end{equation}
  is an invertible matrix. On the other hand, $p$ is a called degenerate
  critical point if the matrix is singular.
\end{defn}

The definition above is independent of local coordinates. For, if $(x^1, ..,
x^k)$ and $(y^1, .., y^k)$ are two local coordinates at $p$, then
\begin{equation}
      \left( \frac{\partial^2 f}{\partial y^k \partial y^l} (p) \right) =
      J^T\left( \frac{\partial^2 f}{\partial x^i \partial x^j} (p) \right)J,
\end{equation}
where $J = \left( \frac{\partial x^i}{\partial y^j}(p) \right)_{i,j}$ is the
coordinate change matrix, which is non-singular.

The non-degeneracy of critical points also has an intrinsic meaning. Let $p$ be
a critical point. We define a symmetric bilinear form $f_{**}$ on $T_PM$, called
the \emph{Hessian} of $f$ at $p$. If $v, w \in T_pM$, then $v$ and $w$ have
extensions $\tilde{v}, \tilde{w}$. Define $f_{**}(v, w) =
\tilde{v}_p((\tilde{w})f)$. We must show that this is symmetric and
well-defined.$f_{**}$ is symmetric, since $\tilde{v}_p((\tilde{w})f) -
\tilde{w}_p((\tilde{v})f) = [\tilde{v}, \tilde{w}_p](f) = 0$\footnote{The last
  equality holds because $p$ is a critical point of $f$.} Therefore $f_{**}(v,
w) = v((\tilde{w})f) = w((\tilde{v})f)$. From both expressions of $f_{**}$ we
see that the definition is independent of the extensions of $v$ and $w$.

If $(x^1, .., x^n)$ is a local coordinate system, then the bilinear form
$f_{**}$ can be expressed as a symmetric matrix $\left(\frac{\partial^2
    f}{\partial x^i \partial x^j}(p)\right)$. Thus $p$ is a non-degenerate
critical point if the Hessian of $f$ at $p$ is a non-degenerate symmetric form.

Let $p$ be a critical point. We define the \emph{index of $f$ at $p$} as the
maximal dimension of subspaces in which $f_{**}$ is negative definite. The
\emph{nullity} is the subspace $V$ of $T_pM$ such that $f_{**}(V,w) = 0$ for all
$w \in T_pM$. The lemma of Morse shows that the behavior of $f$ at $p$ can be
completely described by this index. Before proving this, there is a small lemma:

\begin{lem}
  Let $f$ be a smooth function in a convex neighborhood $V$ of $0$ in $\RR^n$,
  with $f(0) = 0$. Then
  \begin{equation}
    f(x_1, .., x_n) = \sum_{i=1}^{n} x_ig_i(x_1, .., x_n)
  \end{equation}
  for some smooth $g_i$ defined in $V$, with $g_i(0) = \frac{\partial
    f}{\partial x_i}(0)$.
\end{lem}

\begin{proof}
  \begin{equation}
    \begin{aligned}
      f(x_1, .., x_n) &= \int_0^1 \frac{d f(tx_1, .., tx_n)}{dt} dt \\
                     &= \int_0^1 \sum_{i=1}^{n} x_i\frac{\partial f}{\partial
                       x_i}(tx_i) dt
    \end{aligned}
  \end{equation}
  Set $g_i = \int_0^1 \sum_{i=1}^{n} \frac{\partial f}{\partial x_i}(tx_i) dt$.
\end{proof}

\begin{lem}[Lemma of Morse]
  Let $p$ be a non-degenerate critical point for $f$. Then there is a local
  coordinate system $(y^1, .., y^n)$ in a neighborhood $U$ of $p$ with $y^i(p) =
  0$ and for all $i$ and such that the identity
  \begin{equation}
    f = f(p) - (y^1)^2 - .. - (y^\lambda)^2 + (y^{\lambda+1})^2 + .. + (y^n)^2
  \end{equation}
  where $\lambda$ is the index of $f$ at $p$.
\end{lem}

\begin{proof}
  Suppose such a local coordinate system exists. We prove that $\lambda$ must be
  the index of $f$ at $p$. If
  \begin{equation}
        f(q) = f(p) - (y^1(q))^2 - .. - (y^\lambda(q))^2 + (y^{\lambda+1}(q))^2 + .. + (y^n(q))^2,
  \end{equation}
  then
  \begin{equation}
    \frac{\partial^2 f}{\partial y^i \partial y^j} = 
    \begin{cases}
      -2 & i = j \le \lambda, \\
      2  & i = j > \lambda, \\
      0  & otherwise,
    \end{cases}
  \end{equation}
  which shows that $\lambda$ is the index of $f$ at $p$.


  We now show that such a local coordinate system exists.

  
  Let $(x^1, .., x^n)$ be a local coordinate system in a neighborhood $U$ of
  $p$, such that $x^i(p) = 0$ for all $i$. We may also suppose that $f(p) = 0$.
  By the previous lemma we can write
  \begin{equation}
    f(x_1, .., x_n) = \sum_j x_jg_j(x_1, .., x_n)
  \end{equation}
  Since $p$ is a critical point, 
  \begin{equation}
    0 = \frac{\partial f}{\partial x^j}(0) = g_j(0).
  \end{equation}
  
  Applying the lemma to $g_j$, we have
  \begin{equation}
    g_j(x) = \sum_i x_ih_{ij}(x_1, .., x_n).
  \end{equation}
  It follows that
  \begin{equation}
    f(x_1, .., x_n) = \sum_{i,j} x^ix^jh_{ij}(x_1, .., x_n).
  \end{equation}
  
  We may assume $h_{ij}$ is symmetric, since we can write $\bar{h}_{ij} =
  \frac{1}{2}(h_{ij}+h_{ji})$ and get $f = \sum_{i,j}
  x^ix^j\bar{h}_{ij}$. Moreover, $\bar{h}_{i,j}(0) = \frac{1}{2}\frac{\partial^2
    f}{\partial x^i \partial x^j}(0)$, so it is non-singular.

  By induction, we proceed to find the local coordinate system to express $f$ in
  the desired form. Suppose by induction we have coordinates $u_1, .., u_n$ in a
  neighborhood $U$ of $p$ such that
  \begin{equation}
    f = \pm (u_1)^2 \pm .. \pm (u_{r-1})^2 + \sum_{i, j \ge r} u_iu_jH_{ij}(u_1,
    .., u_n),
  \end{equation}
  where $H_{ij}$ is symmetric and non-singular at $0$. After a linear
  change\footnote{If some of the diagonal elements is nonzero, say $H_{qq}$, we
    swap the two coordinates $u_r$ and $u_q$. If all diagonal elements are
    zeros, then we choose $k > r$ such that $\frac{\partial^2 f}{\partial
      u^r \partial u^{k}} \ne 0$. Define new coordinates $(w^1, .., w^n) = (u^1,
    .., u^{r-1}, \frac{u^r+u^k}{2}, u^{r+1}, .., u^{k-1}, \frac{u^r-u^k}{2},
    u^{k+1}, .., u^n)$. Then $\frac{\partial^2 f}{\partial w^r \partial w^r} =
    2\frac{\partial^2 f}{\partial u^r \partial u^k} \ne 0$, as desired.} in the
  last $n-r+1$ coordinates, we may assume that $H_{rr}(0) \ne 0$.Put $g(u_1, ..,
  u_n) = \sqrt{|{H_{rr}}|}$. Then there is a smaller neighborhood $U_2 \subset
  U_1$ of $p$ such that $g \ne 0$. A short calculation shows:

  \begin{equation}
    \begin{aligned}
      \sum_{i,j\ge r}u_iu_jH_{ij} 
      &= (u_r)^2H_{rr} + 2\sum_{i>r}u_iu_rH_{ir}+\sum_{i,j>r}u_iu_jH_{ij} \\
      &= \pm g(u_r+\sum_{i>r}\frac{u_iH_{ir}}{H_{rr}})^2+\sum_{i,j>r}u_iu_jH_{ij}',
    \end{aligned}
  \end{equation}
  where
  \begin{equation}
    H_{ij}' = H_{ij}-\frac{H_iH_j}{H_{rr}}
  \end{equation}
  is symmetric.

  Define new local coordinates
  $v_1$, .., $v_n$ by
  \begin{equation}
    v_i = 
    \begin{cases}
      u_i & i \ne r \\
      g(u_1, .., u_n)\left[u_r+\sum_{i>r}\frac{u_iH_{ir}}{H_{rr}}\right] & i = r
    \end{cases}
  \end{equation}

  By induction, the proof is complete.

\end{proof}

\begin{defn}
  A 1-parameter group of diffeomorphisms of a manifold $M$ is a smooth map
  \begin{equation}
    \phi : \RR \times M \to M
  \end{equation}
  such that
  \begin{itemize}
  \item for each $t \in \RR$ the map $\phi_t : M \to M$ defined by $\phi_t(q) =
    \phi(t, q)$ is a diffeomorphism of $M$ onto itself, and that
  \item for all $t,s \in \RR$, $\phi_{t+s} = \phi_t \circ \phi_s$.
  \end{itemize}
\end{defn}

\begin{lem}
  A smooth vector field on $M$ which vanishes outside of a compact set $K
  \subset M$ generates a unique 1-parameter group of diffeomorphisms on $M$.
\end{lem}

\begin{proof}
  Let $p \in M$. Then there exists a neighborhood $U$ of $p$ and $\epsilon > 0$
  such that the local flow $\phi : (-\epsilon, \epsilon)\times U \to M$ exists
  and is unique and the dependence on initial values is smooth. Since $K$ is
  compact, it can be covered by a finite number of $U$. Let $\epsilon_0$ be the
  smallest $\epsilon$ and set $\phi_t(q) = q$ for all $q \notin K$. Then the
  flow $\phi: (-\epsilon_0, \epsilon_0)\times M \to M$ exists and is unique. If
  $|s|,|t,|s+t| < \epsilon_0$ and set $f(t) = \phi_t(q)$. Then $\frac{df}{dt} =
  X_{f(t)}$ and $f(0) = \phi_s(q)$. Then $f(t) = \phi_{t+s}(q)$ by
  uniqueness. Thus $\phi_{t+s} = \phi_t \circ \phi_s$.

  We can extend the time domain from $(-\epsilon_0, \epsilon_0)$ to $\RR$ by
  define
  \begin{equation}
    \phi_t = \phi_{\frac{\epsilon_0}{2}} \circ \phi_{\frac{\epsilon_0}{2}} \circ
    .. \circ \phi_{\frac{\epsilon_0}{2}} \circ \phi_r,
  \end{equation}
  where $t = k\frac{\epsilon_0}{2}+ r$ and $|r| \le \frac{\epsilon_0}{2}$.
\end{proof}

\section{Homotopy Type in Terms of Critical Values}

Let $M$ be a smooth manifold and let $f: M \to \RR$ be a smooth function without
degenerate critical points. First we define
\begin{equation}
  M^a = f^{-1}(-\infty, a] = \{ p \in M | f(p) \le a \}.
\end{equation}

Then we investigate the ``topological difference'' between $M^a$ and $M^b$,
where $a < b$. The first case is simpler, namely there is no critical point in
$f^{-1}[a,b]$:

\begin{thm}
  Suppose $f^{-1}[a,b]$ is compact and contains no critical points of $f$. Then
  $M^a$ is diffeomorphic to $M^b$. Furthermore, $M^a$ is a deformation retract
  of $M^b$, so that the inclusion map $\iota : M^a \to M^b$ is a homotopy
  equivalence.
\end{thm}

\begin{proof}
  Intuitively, we try to ``squeeze'' $M^b$ into $M^a$ in the direction of the
  gradient of $f$. Choose a Riemannian metric on $M$ and denote the inner
  product as $\langle X, Y \rangle$. The \emph{gradient} of $f$ is characterized
  by the identity\footnote{In fact, it is uniquely defined. To see this, choose
    local coordinates $u^1$, .., $u^n$ and put $X = \frac{\partial}{\partial
      u^i}$ into the identity. Write $grad \: f = f^j\frac{\partial}{\partial
      u^j}$. Then identity becomes $g_{ij}\frac{\partial}{\partial u^j} =
    \frac{\partial f}{\partial u^i}$. Taking inverse of the matrix $(g_{ij})$,
    we get $a^j = g^{ji}\frac{\partial f}{\partial u^i}$.}
  \begin{equation}
    \langle X, grad \: f\rangle = X(f)
  \end{equation}
  
  Let $\rho : M \to \RR$ be a smooth function which is equal to $1/\langle grad
  \: f, grad \: f \rangle$ throughout the compact set
  $f^{-1}[a,b]$\footnote{$grad \: f \ne 0$ in $f^{-1}[a,b]$; otherwise there is
    a critical point. It is easy to check this in terms of normal coordinates.}
  , and which vanishes outside of a compact neighborhood of this set. Then the
  vector field 
  \begin{equation}
    X = \rho (grad \: f)
  \end{equation}
  satisfies the condition in the last lemma of the previous section. Hence $X$
  generates a 1-parameter group of diffeomorphisms
  \begin{equation}
    \phi_t : M \to M.
  \end{equation}

  For fixed $q \in M$ consider the function $t \mapsto f(\phi_t(q))$. Then If
  $\phi_t(q) \in f^{-1}[a,b]$, then
  \begin{equation}
    \frac{d f(\phi_t(q))}{dt} = X_{\phi_t q}(f) = +1
  \end{equation}
  Hence the correspondence $t \mapsto f(\phi_t(q))$ is linear as long as
  $f(\phi_t(q)) \in [a,b]$. It follows that $f(\phi_{t}(q)) = t + f(q)$ in a
  small neighborhood of $0 \in \RR$, where $q \in f^{-1}[a,b]$.

  Consider the diffeomorphism $\phi_{b-a} : M \to M$. By above it's clear
  that $\phi_{b-a}$ maps $M^a$ diffeomorphically onto $M^b$. This prove the
  first part of the theorem.

  Now we construct a 1-parameter family of maps by
  \begin{equation}
    r_t(q) =
    \begin{cases}
      q & if f(q) \le a \\
      \phi_{t(a-f(q))}(q) & if a \le f(q) \le b.
    \end{cases}
  \end{equation}

  Then $r : [0,1] \times M^b \to M^a$ is a homotopy between identity of $M^a$
  and the deformation retract from $M^b$ to $M^a$. This completes the proof.
\end{proof}

The case when $f^{-1}[a,b]$ contains a critical point of $f$ is the essential
part of the Morse theory. It can be shown that $M^b$ is just $M^a$ with a number
of cells attached.

\begin{thm}
  Let $f: M \to \RR$ be a smooth function and let $p$ be a non-degenerate
  critical point with index $\lambda$. Setting $f(p) = c$, suppose that $p$ is
  the only critical point in $f^{-1}[c-\epsilon, c+\epsilon]$, for some
  $\epsilon > 0$. Then for all $\epsilon$ sufficiently small, the set
  $M^{c-\epsilon}$ has the same homotopy type of $M^{c+\epsilon}$ with a
  $\lambda$-cell attached.
\end{thm}

The idea of the proof is to define a new smooth function $F: M \to \RR$ such
that inside a small neighborhood $H$ of $p$, $F < f$, and that $F = f$ outside
$H$. So $F^{-1}[-\infty, c-\epsilon] = M^{c-\epsilon} \cup H$ and
$F^{-1}[-\infty,c+\epsilon] = M^{c+\epsilon}$. If $F^{-1}[c-\epsilon,
c+\epsilon]$ has no critical point and is compact, then $M^{c-\epsilon} \cup H$
is a deformation retract of $M^{c+\epsilon}$ by the preceding theorem. If $H$
can be deformed into a $\lambda$-cell in it, the theorem is proved.

By the idea above, we divide the proof of the theorem into the following
steps.

Choose a coordinate system $u^1$, .., $u^n$ in a neighborhood $U$ of $p$ so that
the identity
\begin{equation}
  f = c - (u^1)^2 - .. - (u^{\lambda})^2 + (u^{\lambda+1})^2 + .. + (u^{n})^2
\end{equation}
holds throughout $U$ (Morse lemma). Then at $p$ we have $u^i(p) = 0$, $i = 1,
.., n$.

Choose $\epsilon > 0$ sufficiently small so that
\begin{itemize}
\item The region $f^{-1}[c-\epsilon,c+\epsilon]$ is compact and contains no
  critical points other than p.
\item The diffeomorphic image of $U$ in $\RR^n$ contains the closed ball
  $\{\sum_i (u^i)^2 \le 2\epsilon\}$.
\end{itemize}

Then $U$ contains a $\lambda$-cell, namely the inverse image of
$\{(u^1)^2+..+(u^{\lambda})^2 \le \epsilon, u^{\lambda+1}=..=u^{n} = 0\}$. Note
that $e^\lambda \cap M^{c-\epsilon}$ is exactly $\{f = c- \epsilon\} =
\{(u^1)^2+..+(u^\lambda)^2 = \epsilon \}$. So that the $\lambda$-cell is really
attached to $M^{c-\epsilon}$.

Next we construct a new smooth function $F : M \to \RR$ as follows. Let $\mu
:\RR \to \RR$ be a smooth function satisfying the conditions
\begin{equation}
  \begin{cases}
    \mu(0) > \epsilon \\
    \mu(r) = 0 \: \forall r \ge 2\epsilon \\
    -1 < \mu'(r) \le 0 \: \forall r.
  \end{cases}
\end{equation}

For example, we can define $\mu$ as follows. Let $\mu = \epsilon'
\frac{\int_x^{2\epsilon} \alpha(t)\alpha(2\epsilon-t)}dt{\int_0^{2\epsilon}
  \alpha(t)\alpha(2\epsilon-t)}dt$, where $2\epsilon > \epsilon' > \epsilon$ and
\begin{equation}
  \alpha(x) =
  \begin{cases}
    0 & x \le 0 \\
    e^{-1/x} & x > 0
  \end{cases}
\end{equation}
is smooth. Then $0 \ge \mu'(r) > -\frac{\epsilon'}{2\epsilon} > -1$.

Now let F coincides with f outside of the coordinate neighborhood $U$, and let
\begin{equation}
  F = f -\mu((u^1)^2+..+(u^{\lambda})^2 + 2(u^{\lambda+1})^2+..+(u^{n})^2)
\end{equation}
within $U$. It is easily verified that $F$ is a well-defined smooth function
throughout $M$.

It is convenient to write $\xi = (u^1)^2+..+(u^{\lambda})^2$ and $\eta =
(u^{\lambda+1})^2+..+(u^{n})^2$.

\begin{asser}
  The region $F^{-1}(-\infty, c+\epsilon]$ coincides with the region $M^{c+\epsilon}$.
\end{asser}

\begin{proof}
  Let $A = f^{-1}[-\infty, c+\epsilon]$, and let $B = F^{-1}[-\infty,
  c+\epsilon]$. If $q \in A$, then $F(q) \le f(q) \le c+\epsilon$ and so $q \in
  B$. To prove the converse, let $q \in B$. If $q \in \{ \xi + 2\eta \le
  2\epsilon \}$, then $f(q) = c - \xi + \eta \le c+ \frac{1}{2}\xi + \eta \le
  c+\epsilon$. If $q \notin \{ \xi + 2\eta \le 2\epsilon \}$, then $f(q) =
  F(q) \le c+\epsilon$. In both cases, $q \in A$.
\end{proof}

\begin{asser}
  The critical points of $F$ are the same as those of $f$.
\end{asser}

\begin{proof}
  \begin{equation}
    \nonumber
    \begin{aligned}
      F &= c - \xi + \eta - \mu(\xi+2\eta) \\
      \frac{\partial F}{\partial \xi} &= -1 - \mu'(\xi+2\eta) < 0 \\
      \frac{\partial F}{\partial \eta} &= 1 - 2\mu'(\xi+2\eta) ge 1 \\
      dF &= \frac{\partial F}{\partial \xi}d\xi+\frac{\partial F}{\partial
        \eta}d\eta
    \end{aligned}
  \end{equation}
  Since the 1-forms $d\xi$ and $d\eta$ are simultaneously 0 only when both
  $\xi$ and $\eta$ equal 0, it follows that $F$ has no critical point in $U$
  other than $p$.
\end{proof}

Now, since $F \le f$, if $q \in F^{-1}[c-\epsilon, c+\epsilon]$, then
$c-\epsilon \le f(q)$. But $q \in F^{-1}(-\infty, c+\epsilon] = f^{-1}(-\infty,
c+\epsilon]$, $f(q) \le c+ \epsilon$. Therefore $q \in f^{-1}(-\infty,
c+\epsilon]$. This proves that $F^{-1}[c-\epsilon, c+\epsilon] \subset
f^{-1}[c-\epsilon, c+\epsilon]$. Being a closed subset of a compact set in a
locally compact Hausdorff space, $F^{-1}[c-\epsilon, c+\epsilon]$ is compact.
Note that $F(p) = c- \mu(0) \le c-\epsilon$. $p \notin F^{-1}[c-\epsilon,
c+\epsilon]$. Together with the preceding theorem we get

\begin{asser}
  $F^{-1}(-\infty, c-\epsilon]$ is a deformation retract of $F^{-1}[-\infty,
  c+\epsilon]$. \qed
\end{asser}

Write $F^{-1}(-\infty, c-\epsilon]$ as $M^{c-\epsilon}\cup H$, where $H$ is the
closure of $F^{-1}(-\infty, c-\epsilon] - M^{c-\epsilon}$. In fact, $H = \{
\epsilon-\mu(\xi+2\eta) \le \xi-\eta \le \epsilon \}$. Define
\begin{equation}
  e^\lambda = \{ q | \xi(q) \le \epsilon, \eta(q) = 0\}
\end{equation}
Then $e^\lambda \subset H$. For, if $q \in e^\lambda$, then $f(q) >
c-\epsilon$. Thus $q \notin M^{c-\epsilon}$. Also, since $\frac{\partial
  F}{\partial \xi} < 0$, $F(q) < F(p) < c-\epsilon$. Thus $q \in F^{-1}(-\infty,
c-\epsilon]$.

\begin{asser}
  $M^{c-\epsilon} \cup e^\lambda$ is a deformation retract of
  $M^{c-\epsilon}\cup H$.
\end{asser}

\begin{proof}
  We operate on the diffeomorphic image on $\RR^n$. In the $\xi-\eta$ plane, $H$
  is the region $F \ge c-\epsilon$ minus the region $f \ge c-\epsilon$.
  
  Denote the homotopy between the retraction and identity by $r_t : [0,1]\times
  (M^{c-\epsilon}\cup H) \to M^{c-\epsilon} \cup e^\lambda$.
  
  Decompose $H$ into 2 regions: (1) $\xi \le \epsilon$ and (2) $\epsilon \le \xi
  \le \eta + \epsilon$.

  In region (1), we simply project $H$ into $e^\lambda$. The homotopy is thus
  given by 
  \begin{equation}
    r_t: (t, (u^1,..,u^n)) \mapsto (u^1, .., u^\lambda, tu^{\lambda+1}, .., tu^n).
  \end{equation}

  We check that the homotopy lies entirely in $H =
  \{\epsilon-\mu(\xi+2\eta) \le \xi-\eta \le \epsilon \}$. It does,
  indeed, since $\epsilon \ge \xi > \xi-t^2\eta > \xi-\eta \ge
  \epsilon-\mu(\xi+2\eta) \ge \epsilon-\mu(\xi+2t\eta)$.

  In region (2), we project this part into $M^{c-\epsilon}$, in the
  $\eta$-direction as the in case (1). Let the following denote the projection:
  \begin{equation}
    (u^1, .., u^n) \mapsto
    (u^1, .., u^\lambda, cu^{\lambda+1}, .., cu^n) \in M^{c-\epsilon}
  \end{equation}
  
  Then it's easy to calculate $c =
  (\frac{\xi-\epsilon}{\eta})^{\frac{1}{2}}$. Then the homotopy is simply the
  line along the projection:
  \begin{equation}
    r_t: (t, (u^1,..,u^n)) \mapsto
    (u^1, .., u^\lambda, s_tu^{\lambda+1}, .., s_tu^n),
  \end{equation}
  where
  \begin{equation}
    s_t = t + (1-t)c.
  \end{equation}

  Again we check that the homotopy lies entirely in $H$. This is true, since
  $s_t \in [0,1]$. More precisely, $\xi-s_t^2\eta \ge \xi-\eta \ge
  \epsilon-\mu$.


  Since $s_t$ is continuous and it is compatible with the case (1) when $\xi =
  \epsilon$, $r_t$ is well-defined and continuous on the union.

  It remains to define $r_t$ on $M^{c-\epsilon}$. We simply define it as the
  identity map. Note that this definition is compatible with that of case (2),
  since on $\xi-\eta=\epsilon$, $s_t = 1$.

  The assertion follows.
\end{proof}

As the last assertion is proved, so is the theorem.

\begin{rmk}
  More generally suppose that there are $k$ non-degenerate critical points
  $p_1$, .., $p_k$ with indices $\lambda_1$, .., $\lambda_k$ in
  $f^{-1}(c)$. Then a similar proof shows that $M^{c+\epsilon}$ has the homotopy
  type of $M^{c-\epsilon}\cup e^{\lambda_1}\cup .. \cup e^{\lambda_k}$.
\end{rmk}

\begin{lem}[Whitehead]
  Let $\phi_0$ and $\phi_1$ be homotopic maps from the sphere $\dot{e}^\lambda$
  to $X$. Then the identity map of $X$ extends to a homotopy equivalence
  \begin{equation}
    k: X \cup_{\phi_0} e^\lambda \to X \cup_{\phi_1} e^\lambda
  \end{equation}
\end{lem}



\section{Manifolds in Euclidean Space: The Existence of Non-degenerate
  Functions}

Let $M$ be a smooth manifold of dimension $k$. By Whitney Embedding Theorem, we
can embed $M$ into $\RR^n, n > k$. Therefore, without loss of generality we may
assume $M \subset \RR^n$. For each $p \in \RR^n$, define a smooth function $L_p
: M \to \RR $ by
\begin{equation}
  L_p(q) = |q-p|^2 = (q-p) \cdot (q-p) = q\cdot q-2q \cdot p + p \cdot p
\end{equation}

We want to find a $p \in \RR^n$ such that $L_p$ has no degenerate critical
points in $M$. Denote $L_p$ as $f$. Let $q \in M$ and let $(u^1, .., u^k)
\mapsto (x^1(u), .., x^n(u))$ be a local coordinate system centered at $q$. Then $q$ is
a critical point if $\frac{\partial f}{\partial u^i}(q) = 0$ for all $i = 1, ..,
k.$; that is,
\begin{equation}
  \frac{\partial \vect{x}}{\partial u^i}(q) \cdot (q-p) = 0.
\end{equation}
It follows that $q \in M$ is a critical point of $L_p$ if and only if $p$ lies
in a line perpendicular to $M$ at $q$. Differentiate again, and we get
\begin{equation}
  \frac{\partial^2 f}{\partial u^i \partial u^j}(q) = 
  2\left(
    \frac{\partial^2 \vect{x}}{\partial u^i \partial u^j}(q) \cdot (q-p) +
    \frac{\partial \vect{x}}{\partial u^i}(q) \cdot \frac{\partial \vect{x}}{\partial u^j}(q)
  \right)
\end{equation}

We get the following conclusions:


$L_p$ is non-degenerate if and only if

\begin{itemize}
\item For each critical point $q$, $p$ lies in a line perpendicular to $M$ at
  $q$.
\item the matrix $\left(
    \frac{\partial^2 \vect{x}}{\partial u^i \partial u^j}(q) \cdot (q-p) +
    \frac{\partial \vect{x}}{\partial u^i}(q) \cdot \frac{\partial \vect{x}}{\partial u^j}(q)
  \right)$
  is non-singular.
\end{itemize}

Thus the main purpose of this section is to establish the existence of such $p
\in \RR^n$.

\subsection{Focal Points}

Let $N = \{ (q,v) \in \RR^{2n} | q \in M, v \in (T_qM)^{\perp} \subset \RR^n \}$
be the normal bundle of $M$ in $\RR^n$. Then $N$ is a manifold of dimension
$k+(n-k)=n$. We define a map $E$ that maps $N$ into the ambient space $\RR^n$ by
\begin{equation}
  E(q, v) = q + v.
\end{equation}
This is a differentiable map between two manifolds of dimension $n$. The point
$q+v$ thus lies in a normal line that is perpendicular to $M$ at $p$. This leads
to the definition of focal points:

\begin{defn}
  $e \in \RR^n$ is a focal point of $(M, q)$ with multiplicity $\mu > 0$ if $e =
  q+v$, where $(q, v) \in N$ and the Jacobian of $E$ at $(q, v)$ has nullity
  $\mu > 0$. $e$ is called a focal point of $M$ if $e$ is a focal point of $(M,
  q)$, $q \in M$.
\end{defn}

Next we fix a $\vect{q} \in M$ and a unit normal vector $\vect{v} \in \RR^n$ and
study the focal points of $(M, q)$ along the line $\vect{l} =
\vect{q}+t\vect{v}$. Before this, we first introduce the second the second
fundamental form of a manifold in $\RR^n$:

\begin{defn}
  Let $\vect{q} \in M$ and let $\vect{v} \in (T_qM)^\perp$, $|\vect{v}| =
  1$. When $g_{ij} = \delta_{ij}$, the second fundamental form\footnote{The more
    general definition is $-(\nabla \tilde{v})^T$, which is a linear
    transformation on $T_qM$. If $(u^1, .., u^k)$ is a normal coordinate system,
    the matrix of the second fundamental from is given by $(-\nabla^T
    \tilde{v})_{ij} = -\frac{\partial \tilde{v}}{\partial u^j} \cdot
    \frac{\partial x}{\partial u^i} = v \cdot \frac{\partial^2 x}{\partial
      u^i \partial u^j}$.} in the direction of $\vect{v}$ is defined as the
  matrix

  \begin{equation}
    \left(
      \vect{v} \cdot \vect{l}_{ij}
    \right)
  \end{equation}
  where $\vect{l}_{ij} = \frac{\partial^2 \vect{x}}{\partial u^i \partial
    u^j}(\vect{q})$.
  The eigenvalues of the matrix is called the principal curvatures $K_1$, ..,
  $K_k$ of $M$ at $\vect{q}$ in the normal direction $\vect{v}$.
\end{defn}

Now comes the key lemma in this section:

\begin{lem}
  The focal points of $(M, q)$ along $\vect{l}$  are precisely the points
  $\vect{q}+K_i^{-1}\vect{v}$, where $1 \le i \le k$, $K_i \ne 0$. Thus there
  are at most $k$ focal points of $(M,q)$ along $\vect{l}$, each being counted
  with its proper multiplicity.
\end{lem}

\begin{proof}
  First we are going to find a local coordinate system of $N$ centered at $(q,
  v)$. Let $\vect{w}_1(u^1, .., u^k)$, .., $\vect{w}_{n-k}(u^1, .., u^k)$ be
  orthonormal vector fields in $N$. Then we have a local coordinate system of
  $N$ at $(q, v)$ defined by
  \begin{equation}
    (u^1, .., u^k, t^1, .., t^{n-k}) \mapsto (\vect{x}(u),
    \sum_{\alpha=1}^{n-k}t^{\alpha} \vect{w}_{\alpha}(u^1, .., u^k)) \in N
  \end{equation}
  Write $E: N \to \RR$ in the coordinates as $E(u, t) = \vect{e}(u, t) \in
  \RR^n$ and get
  \begin{equation}
    \vect{e}(u, t) = \vect{x}(u) + \sum_{\alpha=1}^{n-k}t^{\alpha} \vect{w}_{\alpha}(u^1, .., u^k))
  \end{equation}
  with partial derivatives
  \begin{equation}
    \left\{
      \begin{array}{lcl}
        \frac{\partial \vect{e}}{\partial u^i} & = &
        \frac{\partial \vect{x}}{\partial u^i} + 
        \sum_{\alpha} t^{\alpha}\frac{\vect{w}_{\alpha}}{\partial u^i}  \\
        \frac{\partial \vect{e}}{\partial t^\beta} & = & \vect{w}_\beta
      \end{array}
    \right.
  \end{equation}
  
  Since the n-vectors $\frac{\partial \vect{x}}{\partial u^1}$, ..,
  $\frac{\partial \vect{x}}{\partial u^k}$, $\vect{w}_1$, .., $\vect{w}_{n-k}$
  form a basis of $T_q\RR^n$, the matrix
  \begin{equation}
    A = \left(
      \begin{matrix}
        \left(\frac{\partial \vect{x}}{\partial u^1}\right)^T \\
        \vdots \\
        \left(\frac{\partial \vect{x}}{\partial u^k}\right)^T \\
        \left(\vect{w}_1\right)^T \\
        \vdots \\
        \left(\vect{w}_{n-k}\right)^T
      \end{matrix}
    \right)
  \end{equation}
  is non-singular (we represent a vector as a column matrix). Multiply $A$ with
  the Jacobian $J$ of $E(u, t)$ and we get
  \begin{equation}
    \begin{aligned}
      AJ &= 
      \left(
        \begin{matrix}
          \left(\frac{\partial \vect{x}}{\partial u^1}\right)^T \\
          \vdots \\
          \left(\frac{\partial \vect{x}}{\partial u^k}\right)^T \\
          \left(\vect{w}_1\right)^T \\
            \vdots \\
          \left(\vect{w}_{n-k}\right)^T
        \end{matrix}
      \right)
      \left(
        \begin{matrix}
          \frac{\partial \vect{e}}{\partial u^1} &
          \cdots &
          \frac{\partial \vect{e}}{\partial u^k} &
          \frac{\partial \vect{e}}{\partial t^1} &
          \cdots &
          \frac{\partial \vect{e}}{\partial t^{n-k}} &
        \end{matrix}
      \right) \\
      &=
      \left(
        \begin{matrix}
          \left( \frac{\partial \vect{x}}{\partial u^i} \cdot \frac{\partial
              \vec{x}}{\partial u^j} + \sum_{\alpha}
            t^{\alpha}\frac{\vec{w}_{\alpha}}{\partial u^i} \cdot \frac{\partial
              \vect{x}}{\partial u^j} \right) & 0
          \\
          \sum_{\alpha} t^{\alpha}\frac{\vect{w}_{\alpha}}{\partial u^i} \cdot
          \vect{w}_{\beta} & \id
        \end{matrix}
      \right)
    \end{aligned}
  \end{equation}
  
  Since $J$ has the same rank of $AJ$, $J$ has the same nullity as the upper
  left block of $AJ$. Using the identity
  \begin{equation}
    \begin{aligned}
      0 &= \frac{\partial}{\partial u^i}\left(\vect{w}_{\alpha} \cdot
        \frac{\partial
          \vect{x}}{\partial u^j}\right) \\
      &= t^{\alpha}\frac{\vect{w}_{\alpha}}{\partial u^i} \cdot \frac{\partial
        \vect{x}}{\partial u^j} + \vect{w}_{\alpha} \cdot \frac{\partial^2
        \vect{x}}{\partial u^i \partial u^j},
    \end{aligned}
  \end{equation}
  we see that this upper left hand block is just the matrix 
  \begin{equation}
    \left( g_{ij} - \sum_{\alpha}t^{\alpha}\vect{w}_{\alpha} \cdot \vect{l}_{ij} \right)
  \end{equation}

  We choose the Riemman normal coordinates at $q$, so that $g_{ij}(q) =
  \delta_{ij}$. Now, $\sum_{\alpha}t^{\alpha}\vect{w}_{\alpha} = t\vect{v}$. The
  matrix
  \begin{equation}
    \label{eq:key}
    \left(
      \delta_{ij}-t\vect{v} \cdot  \vect{l}_{ij}
    \right)
  \end{equation}
  is singular if and only if $\frac{1}{t} = K_i$. Furthermore, $\mu$ equals the
  number of distinct $K_i$, each of which counted with its own
  multiplicity. This proves the lemma.
\end{proof}

Suppose $\vect{p} = \vect{q}+t\vect{v}$. Then $\vect{q}$ is a critical point of
$L_p$. Recall the criterion of $q$ being a non-degenerate critical point of
$L_p$. The criterion is exactly Eq (\ref{eq:key}). We thus obtain the
correspondence between the non-degeneracy of $L_p$ and the idea of focal points:

\begin{cor}
  The function $L_p$ is non-degenerate if and only if $p$ is not a focal point
  of $M$. \qed
\end{cor}

$L_p$ is degenerate if and only if $p$ is a focal point, which is a critical
value of a smooth map. By the following theorem of Sard, we see that there are
infinitely many $p \in \RR^n$ such that $L_p$ is non-degenerate:

\begin{thm}[Sard]
  Let $f: M^m \to N^n$ be a smooth map between two smooth manifolds. Then the
  critical values has measure 0 in $N^n$. \qed
\end{thm}

This established the existence of non-degenerate functions on $M$. As a result,
the main theorem in this report is proved:

\begin{thm}
  A differentiable manifold has the homotopy type of a CW-complex.\qed
\end{thm}


The preceding corollary can be sharpened as follows. Let $k \ge 0$ be an integer
and let $K \subset M$ be a compact set.

\begin{cor}
  Any bounded smooth function $f: M \to \RR$ can be uniformly approximated by a
  smooth function $g$ which has no degenerate critical points. Furthermore $g$
  can be chosen so that the i-th derivatives of $g$ on the compact set $K$
  uniformly approximate the corresponding derivatives of $f$, for $i \le k$.
\end{cor}

\begin{proof}
  Choose some embedding $h:M \to \RR^n$ of $M$ as a bounded subset of some
  euclidean space so that the first coordinate $h_1$ is exactly
  $f$. \footnote{By Whitney embedding theorem, there is an embedding $\phi : M
    \in R^n$. Compose each coordinate of $\phi$ with $arctan$, we get a bounded
    embedding, since $\psi = (\tan^{-1} \circ \phi^1, .., \tan^{-1} \circ
    \phi^n)$ is one-to-one and the differential map is injective. Let $h = (f,
    \tan^{-1}\phi)$. Then $h$ is an embedding from $M$ into $\RR^{n+1}$ such
    that $h_1 = f$.} Let c be a large number. Choose a point
  \begin{equation}
    p = (-c+\epsilon_1, \epsilon_2, .., \epsilon_n)
  \end{equation}
  close to $(-c, 0, .., 0) \in \RR^n$ such that the function $L_p: M \to \RR$ is
  non-degenerate. Define
  \begin{equation}
    g(x) = \frac{L_p(x)-c^2}{2c}
  \end{equation}
  A short computation gives
  \begin{equation}
    \begin{aligned}
      g(x) &= \frac{\sum_i h_i^2(x)-h_i(x)p_i+p_i^2-c^2}{2c} \\
           &= f(x) + \frac{\sum_i h_i^2(x)-\sum_i h_i(x)\epsilon_i+\sum_i \epsilon_i^2}{2c}-\epsilon_1.
    \end{aligned}
  \end{equation}
  Since $K$ is compact and each $h_i$ is continuous and bounded, it's easy to
  see that $g(x)$ approximates $f(x)$ uniformly by choosing $c$ large enough and
  $\epsilon_i$ small enough. 
\end{proof}

The preceding theory can also be used to describe the index of the function
$L_p$ at a critical point:

\begin{lem}[Index theorem for $L_p$]
  The index of $L_p$ at a non-degenerate critical point $q \in M$ is equal to
  the number of focal points of (M, q) which lie on the segment from $q$ to $p$;
  each focal point being counted with its multiplicity.
\end{lem}

\begin{proof}
  The index of the matrix 
  \begin{equation}
    \left( \frac{\partial^2 L_p}{\partial u^i \partial
        u^j}(q) \right) = 2(g_{ij}-t\vect{v} \cdot \vect{l}_{ij})
  \end{equation}
  equals the number of negative eigenvalues counted with multiplicity. Recall
  that we can choose the normal coordinates such that $(g_{ij})$ is
  identity. Then$\left( \frac{\partial^2 L_p}{\partial u^i \partial u^j}(q)
  \right)$ has a negative eigenvalue if and only if $1-K_it < 0$ for some
  principal curvatures $K_i$. Thus $K_i > \frac{1}{t}$. Recall that the focal
  points that lies on the line from $q$ through $p$ are of the form
  $\vect{q}+K_i^{-1}\vect{v}$. Therefore the correspondence established.
  
\end{proof}

\section{The Lefschetz Theorem on Hyperplane Sections}

As an application of the previous results, 

\end{document}