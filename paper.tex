\documentclass[12pt]{article}

% This first part of the file is called the PREAMBLE. It includes
% customizations and command definitions. The preamble is everything
% between \documentclass and \begin{document}.

\usepackage[margin=1in]{geometry}  % set the margins to 1in on all sides
\usepackage{graphicx}              % to include figures
\usepackage{amsmath}               % great math stuff
\usepackage{amsfonts}              % for blackboard bold, etc
\usepackage{amsthm}                % better theorem environments
\usepackage{amssymb}


% various theorems, numbered by section

\newtheorem{thm}{Theorem}[section]
\newtheorem{lem}[thm]{Lemma}
\newtheorem{prop}[thm]{Proposition}
\newtheorem{cor}[thm]{Corollary}
\newtheorem{conj}[thm]{Conjecture}
\newtheorem{defn}[thm]{Definition}


\DeclareMathOperator{\id}{id}

\newcommand{\bd}[1]{\mathbf{#1}}  % for bolding symbols
\newcommand{\RR}{\mathbb{R}}      % for Real numbers
\newcommand{\ZZ}{\mathbb{Z}}      % for Integers
\newcommand{\col}[1]{\left[\begin{matrix} #1 \end{matrix} \right]}
\newcommand{\comb}[2]{\binom{#1^2 + #2^2}{#1+#2}}


\begin{document}


\nocite{*}

\title{Algebraic Topology Final Report: Morse Theory}

\author{Shih-Kai Chiu \\
r01221031}

\maketitle

\begin{abstract}
  This report covers roughly the part I of J. Milnor's book(citation). The main
  result is the theorem of Morse: every smooth manifold has the same homotopy
  type of a CW-complex. We prove this theorem by developing the concept of
  non-degenerate smooth functions, their index at critical points and the
  correspondence between the index and the dimension of attaching cells. After
  the main theorem, the existence of non-degenerate smooth functions on
  manifolds is established, and as an application, an alternating proof of the
  Lefschetz hyperplane theorem is given.
\end{abstract}


\section{Basic Definitions and Lemmas}


\subsection{Attaching Cells}

Let $Y$ be any topological space, and let
\begin{equation}
  e^k = \{ x \in \RR^k | \left| x \right| < 1 \}
\end{equation}
be the k-cell. The boundary
\begin{equation}
  \dot{e}^k = \{ x \in \RR^k | \left| x \right| = 1 \}
\end{equation}
is the (k-1)-sphere, $S^{k-1}$. Specifically, we put $e^0$ as a point and set
$\dot{e}^0 = \emptyset$. We call a continuous map $g: S^{k-1} \to Y$ an
attaching map and obtain a new topological space
\begin{equation}
  Y \cup_g e^k
\end{equation}
by identifying each $x \in \dot{e}^k$ with $g(x) \in Y$.


\subsection{Non-degenerate critical points}

Let $M^n$ be a smooth manifold of dimension $n$, and let $f: M \to \RR$ be a
smooth function on $M$. We recall that a point $x \in M$ is called a critical
point of $f$ if the differential map $ df_p : T_pM \to \RR $ is singular. This
is equivalent to $\frac{\partial f}{\partial x^i} (p) = 0$ for all $i = 1,
.. ,k$ and any local coordinate $(x^1, .., x^k)$, since the map $df_p$ in local
coordinate is a row matrix with a non-empty kernel.

\begin{defn}
  (Non-degenerate critical point) Let $p \in M$ and let $(x^1, .., x^k)$ be a
  local coordinate at $p$. $p$ is called a non-degenerate critical point of $f$
  if
  \begin{equation}
    \left( \frac{\partial^2 f}{\partial x^i \partial x^j} (p) \right)
  \end{equation}
  is an invertible matrix. On the other hand, $p$ is a called degenerate
  critical point if the matrix is singular.
\end{defn}

The definition above is independent of local coordinates. For, if $(x^1, ..,
x^k)$ and $(y^1, .., y^k)$ are two local coordinates at $p$, then
\begin{equation}
      \left( \frac{\partial^2 f}{\partial y^k \partial y^l} (p) \right) =
      J^T\left( \frac{\partial^2 f}{\partial x^i \partial x^j} (p) \right)J,
\end{equation}
where $J = \left( \frac{\partial x^i}{\partial y^j}(p) \right)_{i,j}$ is the
coordinate change matrix, which is non-singular.

\section{Homotopy Type in Terms of Critical Values}


\section{Manifolds in Euclidean Space: The Existence of Non-degenerate
  Functions}
Let $M$ be a smooth manifold of dimension $k$. By Whitney Embedding Theorem, we
can embed $M$ into $\RR^n, n > k$. Therefore, without loss of generality we may
assume $M \subset \RR^n$. For each $p \in \RR^n$, define a smooth function $L_p
: M \to \RR $ by
\begin{equation}
  L_p(q) = |q-p|^2.
\end{equation}

We want to find a $p \in \RR^n$ such that $L_p$ has no degenerate critical
points in $M$. Let $q \in M$ and let $(u^1, .., u^k) \mapsto (x^1(u), ..,
x^n(u))$ be a local coordinate at $q$. Then $q$ is a critical point if
$\frac{\partial f}{\partial u^i}(q) = 0 \forall i = 1, .., k.$; that is,


Let $N = \{ (q,v) \in \RR^{2n} | q \in M, v \in (T_qM)^{\perp} \subset \RR^n \}$
be the normal bundle of $M$ in $\RR^n$. Then $N$ is a manifold of dimension
$k+(n-k)=n$. We define a map $E$ that maps $N$ into the ambient space $\RR^n$ by
\begin{equation}
  E(q, v) = q + v.
\end{equation}
This is a differentiable map between two manifolds of dimension $n$. The point
$q+v$ thus line in a normal line that is perpendicular to $M$ at $p$. This leads
to the definition of focal points:

\begin{defn}
  $e \in \RR^n$ is a focal point of $(M, q)$ with multiplicity $\mu > 0$ if $e =
  q+v$, where $(q, v) \in N$ and the Jacobian of $E$ at $(q, v)$ has nullity
  $\mu > 0$. $e$ is called a focal point of $M$ if $e$ is a focal point of $(M,
  q)$, $q \in M$.
\end{defn}

The focal points will play an important role in the construction of a kind





\section{The Lefschetz Theorem on Hyperplane Sections}


\end{document}